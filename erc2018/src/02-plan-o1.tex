\section{Research Plan \todo{B2}}


\subsection{Bridging equality reasoning}

Equality is a fundamental concept in mathematics.
The most basic algebraic structures, such as groups,
are models (in the logical sense) of a set of equational
axioms, the likeness of $x\cdot(y\cdot z) = (x\cdot y)\cdot z$ and $x\cdot 1 = 1\cdot x = x$.
Equality is crucial for automated reasoning, because it means that mathematical objects can have more than one name --- multiple terms may represent the same object.
An inference rule $\infruleshort{K}{P(x),Q(x)}{R(x, x+1)}$
can be applied to derive a goal such as $R(y+z, y+z+1)$,
by instantiating $x\mapsto y+z$;
but it cannot be applied directly to a goal such as
$R(y+1, y+2)$, because it does not (syntactically) match the conclusion pattern $R(x,x+1)$.
The goal first has to be rephrased as the two subgoals
$y + 2 = (y + 1) + 1, R(y+1, (y+1)+1)$, using the so-called \emph{paramodulation} rule%
~\cite{Book2001:Nieuwenhuis},
at which point $K$ can be applied to the second subgoal.
The first subgoal is proved independently using an appropriate axiomatization of the integers.

The paramodulation rule is powerful as it allows the prover to subsitute any term $t_1$ with any other term $t_2$, provided that it can then prove the conjecture $t_1 = t_2$.
This is an extremely useful utility for marking progress in a given proof, but is disastrous from an automated proof search perspective:
Because it can be instantiated with \emph{any} two terms, it opens up an infinite space of possible derivations for any given goal.
Even with a decision procedure that can feasibly find all the possible terms that are equivalent to $t_2$ (and in most cases, such procedure is not available) ---
there would still infinitely many such terms in any non-trivial logic (for instance, $x = x + 0 = x + 0 + 0 = \cdots$).
Careful choice of where and when to apply equality reasoning is essential for effective proof search.

Superposition calculus~\cite{superposition} aims to provide such control over applications of paramodulation
by imposing a \emph{reduction order} $\succ$ over the set of possible terms, and then applying \emph{term rewriting}, replacing $t_1$ by $t_2$ only when $t_1\succ t_2$.
For such a strategy to be effective, $\succ$ most be \emph{confluent}, \ie, if $t\succeq s_1,s_2$, then there
exists $t'$ such that $s_1,s_2 \succeq t'$ (where $t$, $s_{1,2}$, $t'$ are all equivalent).
Otherwise, rewriting may cause proof goals to diverge to
a state where terms occurring in them can no longer be unified via equality reasoning even though they are equivalent.
The widespread solution for constructing such an ordering is to collect the set of all known (universally quantifier) equalities $\Eqs$ and provide them as input to the Knuth-Bendix completion algorithm~\cite{AR1983:Knuth}.
The result is a \emph{directed term rewriting system} that is \emph{normalizing}, that is, any two terms $t_{1,2}$ such that $\Eqs\vdash t_1=t_2$
have a common normal form.
This is done by \emph{orienting} each $x=y \in \Eqs$ to either $x\rwto y$ or $y\rwto x$ based on a Knuth-Bendix ordering of the terms.
(In fact, the ordering is a parameter of the completion algorithm and can be configured.)
It follows that it is sufficient to reduce all the terms in the proof to their normal forms, and then any two equivalent terms will be syntactically equal.

What separates superposition calculus from an ideal solution for equational reasoning is the fact that Knuth-Bendix completion is, in reality, a \emph{semi-algorithm}:
It is not guaranteed to yield a normalizing direction for any set of equations.
Indeed, some inputs exist for which there is no such orientation of the equations.
The problem of telling whether or not a solution exists is itself undecidable (because checking even a TRS for confluence is undecidable).
One glaring situation is the case of the commutative axiom, $x\cdot y = y\cdot x$, where both orientations are equivalent and immediately cause divergence,
leading to the need of instating a side condition $x\succ y$ that arbitatily chooses one order of the operands over the other.
Fine-tuning the term ordering can have drastic effects on whether the Knuth-Bendix algorithm succeeds~\cite{ICRTA2006:Wehrman}.

Equality reasoning in SMT solvers is based on an equality theory solver that uses \emph{congruence closure} to identify (ground) terms that whose equality follows from known (quantifer-free) equality conjectures and from the equality axioms~\cite{JACM1980:Nelson}.
An efficient implementation is achieved through use of \emph{equality graphs} --- \emph{e-graphs} --- that allow fast computation of congruence closure~\cite{Thesis1980:Nelson}.
E-graphs where also in the core of one of the earlier ATPs, Simplify~\cite{simplify}.

\begin{proposal}E-graphs are a suitable formalism to serve as the bridge that unifies equality reasoning in SMT and ATP.
\end{proposal}

We propose to build a framework for equality reasoning based on e-graphs and term rewriting systems (TRSs), 
inspired by similar uses that came up in compiler optimizations and program synthesis~\cite{everything-zack}.
We will use egg~\cite{egg}, a collection of state-of-the-art algorithms for manipulating e-graphs and for fast rewriting over the e-graph representation.
In an e-graph, terms are grouped in equivalence classes (e-classes), and shared sub-terms are not duplicated, leading to a very compact representation.
The use of TRSs provides a way to handle universally-quantified equality statements such as $\forall x,y,z. x\cdot(y\cdot z) = (x\cdot y)\cdot z)$ and $\forall x,y. x\cdot y = y \cdot x$.
Previous work in compilers relied on the concept of \emph{equality saturation}~\cite{equsat}:
That term rewriting can be applied to the e-graph until it eventually captures all possible consequences of the underlying equalities, and further applications of the rewrite rules would not contribute any new terms ---
the e-graph is saturated.
Naturally, this cannot be achieved in all circumstances.
Our first task is therefore going to be a theoretical one, investigating the property of saturation.

\begin{researchquestion}What characterizes a TRS that is \emph{saturating}, that is, reaches saturation on any input e-graph?
\end{researchquestion}
 
This question is a lifting of the finite termination and confluence properties of TRSs when operating on standard terms.
It is trivial to see that a finitely-terminating TRS will also be saturating,
although there can be an exponential blowup in the number of steps required for termination due to exploration of all possible rewrites in the e-graph setting.
The other direction is definitely not true:
We already saw the case of $x \cdot y \rwto y\cdot x$ as an example to a non-terminating rule.
This rule is, in fact, saturating, because any $\cdot$ term has exactly two variants.
Similarly, $x \rwto 1\cdot x$ is even \emph{diverging}, in the sense that every application of it yields a new term that was not derived before ($1 \to 1\cdot 1 \to 1\cdot 1 \cdot 1 \to \cdots$);
however, thanks to the compact representation of e-classes, infinitely many terms can be represented in a single, finite e-graph, so this rule is also saturizing.
This is not to say that all RTSs are saturating w.r.t. e-graphs.
What makes some RTSs saturizing while others diverge?
Can we 