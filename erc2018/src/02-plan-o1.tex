\section{Research Plan \todo{B2}}


\subsection{Bridging equality reasoning}

Equality is a fundamental concept in mathematics.
The most basic algebraic structures, such as groups,
are models (in the logical sense) of a set of equational
axioms, the likeness of $x\cdot(y\cdot z) = (x\cdot y)\cdot z$ and $x\cdot 1 = 1\cdot x = x$.
Equality is crucial for automated reasoning, because it means that mathematical objects can have more than one name --- multiple terms may represent the same object.
An inference rule $\infruleshort{K}{P(x),Q(x)}{R(x, x+1)}$
can be applied to derive a goal such as $R(y+z, y+z+1)$,
by instantiating $x\mapsto y+z$;
but it cannot be applied directly to a goal such as
$R(y+1, y+2)$, because it does not (syntactically) match the conclusion pattern $R(x,x+1)$.
The goal first has to be rephrased as the two subgoals
$y + 2 = (y + 1) + 1, R(y+1, (y+1)+1)$, using the so-called \emph{paramodulation} rule%
~\cite{Book2001:Nieuwenhuis},
at which point $K$ can be applied to the second subgoal.
The first subgoal is proved independently using an appropriate axiomatization of the integers.

The paramodulation rule is powerful as it allows the prover to subsitute any term $t_1$ with any other term $t_2$, provided that it can then prove the conjecture $t_1 = t_2$.
This is an extremely useful utility for marking progress in a given proof, but is disastrous from an automated proof search perspective:
Because it can be instantiated with \emph{any} two terms, it opens up an infinite space of possible derivations for any given goal.
Even with a decision procedure that can feasibly find all the possible terms that are equivalent to $t_2$ (and in most cases, such procedure is not available) ---
there would still infinitely many such terms in any non-trivial logic (for instance, $x = x + 0 = x + 0 + 0 = \cdots$).
Careful choice of where and when to apply equality reasoning is essential for effective proof search.