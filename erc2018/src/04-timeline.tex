\subsection*{Work Plan}

\newcommand\yrsz{5mm}
\newcommand\doh{\cellcolor{blue!25!white}}
\newcommand\dog{\cellcolor{gray!25!white}}

\begin{tabular}
{|l|p{\yrsz}p{\yrsz}|p{\yrsz}p{\yrsz}|p{\yrsz}p{\yrsz}|p{\yrsz}p{\yrsz}|p{\yrsz}p{\yrsz}|}
\hline
Task/Date & Y1 & & Y2 & & Y3 & & Y4 & & Y5 & \\ \hline
\textbf{Objective 1:} \textit{Bridging equality reasoning} & & & & & & & & & & \\
RQ1: Characterization of TRS saturation & \doh & \doh & \doh & & & & & & & \\
RQ2: E-graph coloring & & & \doh & \doh &&&&&& \\
RQ3: Unification modulo theories & & & & \doh & \doh & \doh &&&& \\ \hline
\textbf{Objective 2:} \textit{Bridging induction} &&&&&&&&&& \\
RQ4: TC axiomatization \& derivation & \doh & \doh &&&&&&&& \\
RQ5: Cyclic TC proof search & & \doh & \doh & \doh &&&&&& \\
RQ6: Inductive-deductive hybrid & & & & \doh & \doh & \doh &&&& \\ \hline
\textbf{Objective 3:} \textit{Modular reasoning with lemmas} & & & & & & & & & & \\
RQ7: Usefulness metric & & & & & & \doh & \doh & & & \\
RQ8: Untangle-and-conquer  & & & & & & & \doh & \doh & \doh & \\ \hline
\textbf{Objective 4:} \textit{Extrapolating to software}  & \multicolumn{2}{c|}{\color{gray}\small (ongoing)} & \dog & \dog & \dog & \dog & \dog & \dog & \dog & \dog \\
RQ9: SL-TC synergy & & & \doh & \doh &&&&&& \\
BM1: CertiKOS (OS kernel) & & & & & \doh & \doh &&&& \\
BM2: DataCert (DBMS+SQL) & & & & & & \doh & \doh &&& \\
BM3: CompCert (C compiler) & & & & & & & \doh & \doh & \doh &  \\
BM4: Galois (electronig voting) & & & & & & &  & & \doh & \doh  \\
\hline
\end{tabular}

\subsubsection*{Project Team}
The work will be performed by the PI, a postdoc, two Ph.D. students, one M.Sc. student, and a software engineer.
All students will be dedicated to the project full time.
The engineer will work at 25\%, which is the estimated part of the total work time in which an engineer's assistance will contribute to the project.
The postdoc's role in the team would be to learn and pass along the multiple formalisms and calculi needed for Objectives 1,2,3, with an emphasis on Objective 2 where deep understanding both of cyclic proof systems and of IC3/PDR is crucial for a successful integration,
and there is a large amount of background material to process.
The postdoc will take an active part in this integration and will work directly on RQ5,6.
The two Ph.D.s will work mainly on Objectives 1 and 3, and in particular, will work together on equality reasoning, shifting to modular reasoning when equality reasoning has yielded a substantial collection of lemmas for untangle-and-conquer to work on top of.
One of the Ph.D.s will work on uses of TC logic, in which RQ4 is followed in RQ9.
For the evaluation part, assignment of benchmarks to researchers will be done based on the more essential pieces of reasoning per case,
which is itself part of the research enveloped in Objective 4.
An M.Sc. student will help with investigating existing proofs and developments from DeepSpec and locate others, possibly newer benchmarks --- which they will then discuss with the group so that the PI and the rest of the team can decide on the best distribution of labor.
The role of the engineer will come into play as our prototypes become more mature, and the team will produce polished, maintained tools that will be provided as open-source for use by the research community.

\subsubsection*{Existing Resources}

The Host Institute provides the PI and his team with lab space and surrounding facilities.
Our lab is equipped with two Intel Xeon-based Linux servers for running compute-intensive experiments, which are essential when dealing with enumerative search.
Additional institute funds will enable the purchase of GPUs, and those could be harnessed to the benefit of the project as well.
The CS Faculty's IT department will assist in server maintenance and installation of GPUs and accompanying software.