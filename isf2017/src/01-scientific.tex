\section{Scientific Background}

It is definitely the Age of Software.
After having infiltrated our homes, our cars, our pockets and wrists,
after startup engineers have seemingly launched every app imaginable,
the demand for software refuses to decline.
Its success in the private space catalyzes its application to the public
space as well: clinics and offices shift toward Web-based appointment
scheduling and information delivery, and even government agencies are
inclined to offer their services online.
The thirst for information causes more data stores to open up to the
public, and the flow of data requires organization, searching and sorting
to make sense of it.
Soon enough our wallets will be filled with electronic currency and
manage trade contracts for us.
In this new economy it will become hard to imagine we could ever
shop for groceries without software.

To cope with ever-increasing need to create more software, developers
will have to adopt new practices.
Today's development tools limit code reuse and collaboration in larger
scales, \eg in an open source community setting.
Software modules are being rewritten and old code ages quickly and becomes
unusable due to rapidly changing APIs, unless it is routinely maintained,
incurring high overheads.
Over time, code bases become harder to understand, maintain, and reuse.
Accompanying design documents and other documentation are written, but
they are hard to keep up-to-date with the code, and at any rate, it does
not seem that the right way to understand a computer program is to read
a book about it.

\begin{subparagraph}{Clearly distinguishing semantics from implementation}
Most software starts its life as clean, elegant code, thoughtfully designed.
What happens then is that concerns such as portability, performance,
and integration with other software arise, and the developers are forced
to revise their code.
Code that has been subject to many changes is notorious for becoming
obfuscated, and also starts to deviate from the original design and
documentation.
A particular syndrome we wish to address is that the \emph{meaning} of
the software, often called its \emph{semantics}, sometimes \emph{formal specification},
and colloquially \emph{"The What"}, is intertwined and mixed with all
other aspects of programming within the code --- its \emph{implementation},
\emph{"The How"}~\cite{CACM2012/Bodik}.
Creating software layers for software semantics and its implementation,
in conjunction with a corresponding compiler that knows about the connection
between the two, can help manage the risks from changes by
\begin{enumerate*}[label=(\textit{\roman*})]
\item only changing the
specification when requirements actually change, 
\item keeping the specification smaller than the whole program, making it easier to review, and
\item making sure, mechanically, that the correlation between the layers is maintained.
\end{enumerate*}
\end{subparagraph}

\begin{subparagraph}{Not repeating oneself}
The previous point is very much addressed by software engineering approaches
like design-by-contract and by formal methods such as model checking and Hoare-Floyd
style verification.
What these techniques generally lack is the transference of information across
abstraction layers:
after writing a contract, a programmer has to provide an implementation,
which repeats many of the details of the contract in code form;
after writing a procedure, to verify it, a programmer has to provide a formal
specification in some logic, repeating many of the implementation details.
To avoid repetitiveness, some form of synthesis is required, backed up by some
form of formal reasoning that tells the system which parts can be re-used when
refining and which parts change and need to be specified.
Kuncak \etal~\cite{CACM2012/Kuncak} argue that the task resembles that of
constructing a specialized \emph{constraint solver} for a set of constraints
at hand; this is in contrast with generic constraint solvers, which are
either not available for a particular domain (due to inherent undecidability)
or are not efficient enough.
\end{subparagraph}

\begin{paragraph}{Related Work}
Work on semi-automated programming began quite long ago at Kestrel Institute~\cite{AI1985/Smith},
which led to the development of several prototype projects,
KIDS~\cite{TSE1990/Smith}, DTRE~\cite{CPS1991/Blaine}, 
and Specware~\cite{2001/McDonald:specware}.
These systems are built on the concept of an algorithmic knowledge base encoded
as a library, and matched against program specifications to apply the
right transformation and obtain a correct-by-construction implementation.
KIDS contains one of the early interactive proof assistants, to support
fulfillment of pre-conditions required for some algorithmic solutions.

The same concepts inspired the creation of the Refinement Calculator~\cite{TPHOLs1996/Butler};
it provides an interactive environment where expressions are replaced with
new ones that subsist in some transitive relation, such that multiple steps
can be combined to a proof that the same relation holds between the first
revision and the last.
The relation can be some order on values ($\leq$), or a more complicated relation
on programs such as \emph{behavioral refinement}, in which the proposed substitution
exhibits the same set of run-time behaviors or fewer compared to the original
term.
This asserts that any safety property satisfied by the original program is held
on the revised.

AutoBayes~\cite{JFP2003/Fischer} is an example for a domain-specific automatic programming system
developed by NASA.
It contains a large database of statistical algorithms for constructing and
analyzing models.
A solver tries different \emph{schemas}, allowing backtracking to find
a computationally optimal combination, and derives a specialized algorithm
for learning a specific model supplied by the user.

In Scala\textsuperscript{Z3}, similar guidelines were followed by match-making
constraints written in the Scala programming language with a palette of
solving techniques encoded in first-order logic and implemented on top of
Z3, a leading SMT solver. SMT (Satisfiability Modulo Theory) is a term used
to describe a family of solvers that can check satisfiability of a logical
formula under some intended interpretation of the domain (\eg natural numbers).
Such solvers have been used excessively for verification in the past,
but less so for synthesis purposes.

A more recent system, Fiat~\cite{POPL2015/Delaware}, is what can be described
as a glorified domain-specific language compiler.
Instead of the usual code generation--optimization phases, it employs a set
of domain tactics that transform the input program, and then use an extraction
mechanism implemented in Coq to generate the code.
Transformations are done in a high-level theory, but a library for creating
tactics allows a language designer to describe implementation-specific details
that are crucial for some performance benefits.

Generally speaking, early approaches to transformational synthesis relied on
manual guidance, while automation was mainly limited by available computational
power. Later systems allow for greater automation in exploring the program space
but inhabit domain niches, since the general problem is still considered computationally hard.

The concept of refinement as used by \cite{TPHOLs1996/Butler,POPL2015/Delaware} mentioned above
is not constrained to synthesis, but has applications in formal verification
as well. In \cite{CAV2015/Hawblitzel}, researchers from Microsoft Research laboratories
show that abstracting concurrent code into several layers can greatly simplify the task of generating
a machine-checkable correctness proof.
While it takes advantage of SMT technology, this approach does require a rather
large amount of annotations and ghost code.
Domain-specific abstractions were also shown useful in analysis and,
surprisingly, debugging, by Bod\'ik \etal~\cite{SNAPL2107/Bodik}.

\emph{Interactive} synthesis, not necessarily deductive,
is trending, and several papers have been disseminated on the subject
\cite{SYNASC2012/Gulwani,CACM2012/Gulwani,CAV2011/Gvero,Onward2015/Barman}.
These revolve mostly around programming by demonstration or via examples.
Still, syntax plays a role~\cite{NATOSec2015/Alur} and the user can restrict the resulting program
to a particular API or control structure, thus guiding the search.
\end{paragraph}
