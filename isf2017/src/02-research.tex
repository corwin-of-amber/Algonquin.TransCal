\section{Research Objective and Expected Significance}

This research aims to improve the productivity of software development
as well as the quality of the developed code by promoting a new style
of \emph{gradual programming by refinement}.
% explain why refinement is great etc.

\begin{paragraph}{Challenges}
The process becomes more interactive with the programmer and the
compiler taking turns. Correctness should be preserved at all times
to deliver useful abstractions.
Smaller steps allow more interactivity but require more programmer
effort; large steps increase automation but create higher latency
and may not generalize well.
\end{paragraph}

% sub paragraph
\begin{paragraph}{Separation of Concerns}
Effective abstractions simplify the thought process of putting together
a piece of software by ignoring some of the aspects of it, focusing on
one concern at a time. This allows the developer to filter out background
noise and pay attention to all the different functionality requirements.
Previous approaches to \emph{aspect-oriented programming} recognized this
difficulty and were able to separate away e.g. the logging layer or error
handling policy of the application. Their biggest shortcomings were 
(i) being overly attached to the language syntax rather than semantics, 
(ii) providing a rather thin white-box interface between modules
    (so-called \emph{aspects}), and
(iii) committing to the orthodox batch-compilation development cycle.
To overcome these obstacles, we have to define a richer meta-programming
language that can reason about computation at the semantic level.
\end{paragraph}