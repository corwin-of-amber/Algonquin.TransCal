\section{Preliminary Results}

\begin{paragraph}
{Bellmania.} As a demonstration of the efficacy of synthesis in non-trivial
scenarios and as a response to a challenge introduced by Prof. Charles Leiserson
at MIT,
I designed an developed a formal framework for mechanized derivation of 
cache-oblivious implementations for sequential and parallel 
dynamic programming algorithms. Dynamic programming is pervasive in several 
applications of computer science and can greatly speed up otherwise 
intractable computations, but requires large amounts of memory, 
which puts stress on memory access bandwidths and creates performance 
bottlenecks. These problems can be alleviated by more effective utilization 
of the machine’s fast memory caches, 
but doing so requires complex reordering of memory accesses to achieve 
desired spatial and temporal locality properties. 
By building up on recent developments in parallel high-performance computing, 
in particular recursive divide-and-conquer applied to dynamic programming,
and by restating it as a refinement problem that starts from a na\"ive, 
inefficient loop implementation and targets a better, 
equivalent implementation, which is also cache-oblivious. 
I encoded the concepts underlying the divide-and-conquer technique as program 
transformation steps set up in an interactive environment where the user chooses 
which transformations to apply and the system continuously verifies semantic 
soundness of each step. 
This allowed me to produce verified implementations of several dynamic programming 
algorithms using a handful of transformation rules.
\end{paragraph}
