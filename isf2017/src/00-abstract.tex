% This has to be a separate document according to the submission format.
\documentclass[11pt]{article}

\usepackage{setspace}
\usepackage[margin=1.25in]{geometry}

\onehalfspacing

\pagestyle{empty}

\begin{document}

\hfill
\begin{tabular}{r}
Application No. 582/18 \\
PI1 Name: Shachar Itzhaky
\end{tabular}

\medskip\noindent
{%
\fontsize{14pt}{14pt}\selectfont%
Scientific abstract --- High-level programming through refinement and program derivation
}

\medskip
Sixty years after the creation of the first commercial compiler,
computer programs are written using such rudimentary tools as a text
editor and a console.
While IDEs are offering gradually increasing levels of comfort in the
editing task itself, such as syntax highlighting, quick navigation,
and context-sensitive completion, non of the features have an impact
like that of the Fortran compiler had on programming.
As the sheer amount of software code grows and its complexity rapidly
exceeds the capacity of its developers, a new type of tools and techniques
is gravely needed.
Several attempts were made in the past to develop automated programming
tools and higher-level languages, but were lacking foundation to
reach the level of usability expected from a day-to-day-use environment.
The area of formal methods has true potential to fill this need, as
recent advances have made the analysis of large and complex software
possible, as well as the creation of machine-checkable proofs for
non-trivial mathematical theorems, indeed in some cases where manually
writing and checking the proofs could take years.
This raises the bar for the level of certainty we can expect from
computer programs.
While software verification practices alone can improve code quality
and reliability, they incur high development overheads and are therefore
rarely used, except in mission-critical systems.
Software synthesis, on the other hand, carries a promise to \emph{decrease}
programmer effort while giving similar guarantees.
The combination of formal methods and synthesis is natural, and interest
within the programming languages community is rising.

This research explores a variety of opportunities to discover new methods
based on formal methods and verification, targeted toward usable synthesis.
A key feature of proposed solutions is interactivity, that is, ongoing
involvement of the human developer in the decision making process,
the prospects of which are twofold:
it alleviates inherent computational complexity in the synthesis procedure,
and also gives the developer more control of the process and feedback for
``what went wrong''.
Humans are experts at decomposing problems into sub-tasks, making solving
them algorithmically faster by an exponential factor;
computers are good at handling large sets of small details, exploring
corners that a human might overlook.
Finding the right tasks to automate and the type of feedback to provide
is crucial at reaching a successful, productive interaction.

\end{document}
